\documentclass{article}

\usepackage{bm, bbm, amsmath}
\usepackage{amssymb,stmaryrd}
\usepackage{color}
\usepackage{hyperref}
\usepackage[american]{babel}
\usepackage[utf8]{inputenc}
\usepackage[OT1]{fontenc}

\usepackage{algorithm}
\usepackage{algorithmic}
\renewcommand{\algorithmicrequire}{\textbf{Input:}}
\renewcommand{\algorithmicensure}{\textbf{Output:}}

\def\Q {\ensuremath{\mathbb{Q}}}
\def\Z {\ensuremath{\mathbb{Z}}}
\def\F {\ensuremath{\mathbb{F}}}
\def\Tr {\ensuremath{\mathrm{Tr}}}

\def\M {\ensuremath{\mathsf{M}}}

\newcommand{\todo}[1]{\textcolor{red}{TODO: #1}}

\newtheorem{Def}{Definition}
\newtheorem{Theo}{Theorem}
\newtheorem{Prop}{Proposition}
\newtheorem{Lemma}{Lemma}

\title{Computing isomorphisms and embeddings of finite fields}
\author{Ludovic Brieulle, Luca De Feo, Javad Doliskani,\\ Jean Pierre
  Flori and Éric Schost}


\begin{document}

\maketitle
\begin{abstract}
  Let $\F_q$ be a finite field.  The finite field isomorphism problem
  asks, given two irreducible polynomials $f,g$ in $\F_q[X]$ with
  $\deg f$ dividing $\deg g$, to compute an explicit description of a
  field embedding of $\F_q[X]/f$ into $\F_q[X]/g$. When $\deg f = \deg
  g$, this is also known as the isomorphism problem.
  
  We review classical algorithms for the two problems, and compare
  their asymptotic complexities. We also propose new improvements and
  generalizations to the algorithms, implement them, and compare them
  with the state of the art.
\end{abstract}

%%%%%%%%%%%%%%%%%%%%%%%%%%%%%%%%%%
%%%%%%%%%%%%%%%%%%%%%%%%%%%%%%%%%%

\section{Introduction}

Let $q$ be a prime power and let $\F_q$ be a field with $q$
elements. Let $f$ and $g$ be irreducible polynomials in $\F_q[X]$,
with $\deg f$ dividing $\deg g$. Define $k=\F_q[X]/f$ and
$K=\F_q[X]/g$, then there is an embedding $\varphi:k\hookrightarrow
K$, unique up to $\F_q$-automorphisms of $k$. The goal of this paper
is to describe algorithms to efficiently represent and evaluate one
such embedding.

\todo{Motivation, previous work.}

All the algorithms we are aware of, split the embedding problem in two
sub-problems:
\begin{enumerate}
\item Determine elements $\alpha\in k$ and $\beta\in K$ such that
  $k=\F_q[\alpha]$, $K=\F_q[\beta]$, and such that there exists an
  embedding $\varphi$ mapping $\alpha\mapsto\beta$. We refer to this
  problem as the \emph{Embedding description}.
\item Given elements $\alpha$ and $\beta$ as above, given $\gamma\in
  k$ and $\delta\in K$, solve the following problems:
  \begin{itemize}
  \item Compute $\varphi(\gamma)\in K$.
  \item Test if $\delta\in\varphi(k)$.
  \item If $\delta\in\varphi(k)$, compute $\varphi^{-1}(\delta)\in K$.
  \end{itemize}
  We refer collectively to these problems as the \emph{Embedding
    evaluation}.
\end{enumerate}

\paragraph{Complexity notation} We measure all complexities in number
of operations $+$, $\times$, $\div$ in $\F_q$. We write $\M(n)$ for
the cost of multiplying two polynomials in $\F_q[X]$ of degree at most
$n$.


\subsection{Naive algorithms}

\todo{Describe polynomial factorization}

\todo{Describe linear algebra}


%%%%%%%%%%%%%%%%%%%%%%%%%%%%%%%%%%
%%%%%%%%%%%%%%%%%%%%%%%%%%%%%%%%%%

\part{Embedding description}

%%%%%%%%%%%%%%%%%%%%%%%%%%%%%%%%%%

\section{Lenstra's and Allombert's algorithms}

\subsection{Faster variant}


%%%%%%%%%%%%%%%%%%%%%%%%%%%%%%%%%%

\section{Rains' algorithm}

\subsection{Gaussian periods}

\subsection{Rains' special algorithm}

\subsection{Rains' cyclotomic algorithm}

\subsection{Complexity analysis}


%%%%%%%%%%%%%%%%%%%%%%%%%%%%%%%%%%

\section{Elliptic Rains' algorithm}

\subsection{Elliptic periods}

\subsection{Elliptic algorithm}

\subsection{Complexity analysis}


%%%%%%%%%%%%%%%%%%%%%%%%%%%%%%%%%%

\section{Experimental results}

%%%%%%%%%%%%%%%%%%%%%%%%%%%%%%%%%%
%%%%%%%%%%%%%%%%%%%%%%%%%%%%%%%%%%

\part{Embedding evaluation}

%%%%%%%%%%%%%%%%%%%%%%%%%%%%%%%%%%

\section{Algorithms specific to normal bases}

%%%%%%%%%%%%%%%%%%%%%%%%%%%%%%%%%%

\section{Monomial-dual bases pairs}

%%%%%%%%%%%%%%%%%%%%%%%%%%%%%%%%%%

\section{Experimental results}

%%%%%%%%%%%%%%%%%%%%%%%%%%%%%%%%%%

\section{Conclusion}

%%%%%%%%%%%%%%%%%%%%%%%%%%%%%%%%%%
%%%%%%%%%%%%%%%%%%%%%%%%%%%%%%%%%%

\bibliographystyle{plain}
\bibliography{defeo}

\end{document}


% Local Variables:
% ispell-local-dictionary:"american"
% End:

%  LocalWords:  isomorphism
