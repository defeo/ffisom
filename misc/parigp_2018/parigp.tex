\documentclass[francais]{beamer}

\usepackage[utf8]{inputenc}
\usepackage[english]{babel}

\usepackage{wrapfig}
\usepackage{graphicx}
\graphicspath{{graphics/}}
\usepackage{tikz}
\usetikzlibrary{matrix,arrows,arrows.meta,positioning}
\tikzset{>/.tip={Computer Modern Rightarrow[scale=2,line width=1pt]}}
\tikzset{|/.tip={Bar[scale=2,line width=1pt]}}
\tikzset{c_/.tip={Hooks[scale=2,line width=1pt,right]}}
\tikzset{c^/.tip={Hooks[scale=2,line width=1pt,left]}}

\colorlet{alert}{red}

\def\Q {\ensuremath{\mathbb{Q}}}
\def\Z {\ensuremath{\mathbb{Z}}}
\def\F {\ensuremath{\mathbb{F}}}
\def\Tr {\ensuremath{\mathrm{Tr}}}
\def\M {\ensuremath{\mathsf{M}}}
\def\tildO {\tilde{O}}
\def\Id {\ensuremath{\mathrm{Id}}}

\DeclareMathOperator{\Gal}{Gal}
\DeclareMathOperator{\ord}{ord}

\newcommand{\paragraph}[1]{\smallskip\textbf{#1}}

\makeatletter
\def\beamer@andinst{ }
\makeatother
\title[Lattices of finite fields]{Algorithms for lattices of compatibly embedded finite fields}
\author[LB, LDF, JD, JPF, ÉS]{Luca De Feo\inst{2} \and Jean-Pierre~Flori\inst{4}\\
  \bigskip
  \small joint work with\\
  Ludovic Brieulle\inst{1} \and  Javad Doliskani\inst{3}\\
  \and Édouard Rousseau\inst{2} \inst{5} \and Éric Schost\inst{3}}
\institute[UAM, UVSQ, UW, ANSSI]{\tiny \vspace*{0.5em}\inst{1} Université d'Aix-Marseille \and
  \inst{2} Université de Versailles -- Saint-Quentin-en-Yvelines\\ \and
  \vspace*{-0.8em} \inst{3} University of Waterloo \and
  \inst{4} Agence nationale de sécurité des systèmes d'information\\
  \inst{5} Télécom Paristech}
\date[2018/01/17]{}

\begin{document}

\begin{frame}<handout:0> \titlepage
\end{frame}

\section{The embedding problem}

\begin{frame}\frametitle{The embedding problem}
%  \begin{columns}
%    \begin{column}{.3\textwidth}
   \begin{wrapfigure}{r}{0.45\textwidth}
     \centering
 %     \resizebox{\textwidth}{!}{
      \begin{tikzpicture}[node distance=8em,font=\large]
        \node(Fq){$\F_q$};
        \node(Fqf)[above left of=Fq]{$k=\F_q[X]/f(X)$};
        \node(Fqg)[above right of=Fqf]{$K=\F_q[Y]/g(Y)$};
        \draw[]
        (Fq) edge (Fqf)
        (Fq) edge (Fqg)
        (Fqf) edge[left,auto,color=alert] node{$\varphi$} (Fqg);
      \end{tikzpicture}
  %  }
    \end{wrapfigure}
  %  \end{column}
  %  \begin{column}{0.7\textwidth}
   % \begin{definition}

    \paragraph{Let}
    \begin{itemize}
    \item $\F_q$ be a field with $q$ elements,
    \item $f$ and $g$ be irreducible polynomials in $\F_q[X]$ and
      $\F_q[Y]$,
    \item $r=\deg f$, $s=\deg g$ and $r|s$.
    \end{itemize}
  %\end{definition}
  %\begin{theorem}
    \vfill
    There exists a field
    embedding \[\color{alert}\varphi:k\hookrightarrow K,\] unique up to
    \mbox{$\F_q$-auto}morphisms of $k$.
%  \end{column}
%\end{columns}
% \end{theorem}
\end{frame}

\begin{frame}\frametitle{Embedding description}
      \begin{wrapfigure}{r}{0.4\textwidth}
      \centering
      \begin{tikzpicture}[node distance=8em,font=\Large]
        \node(Fq){$\F_q$};
        \node(Fqf)[above left of=Fq]{$k=\F_q[\alpha]$};
        \node(Fqg)[above right of=Fqf]{$K$};
        \begin{scope}[alert,node distance=2em]
          \node(a)[above of=Fqf]{$\alpha$};
          \node(b)[left of=Fqg]{$\beta$};
        \end{scope}
        \draw[auto,font=\small]
        (Fq) edge node[left]{$r$} (Fqf)
        (Fq) edge node[right]{$s$} (Fqg)
        (Fqf) edge (Fqg);
        \draw[|->,alert]
        (a) edge (b);
      \end{tikzpicture}
    \end{wrapfigure}

    \textbf{Determine} elements $\alpha$ and $\beta$ such
    that
    \begin{itemize}
    \item $\alpha$ generates $k=\F_q[\alpha]$,
    \item there exists $\varphi:\alpha\mapsto\beta$.
    \end{itemize}
    \vfill
    \paragraph{Naive solution:} take
    \begin{itemize}
    \item $\alpha= X \mod f(X)$, and
    \item $\beta$ a root of $f$ in $K$.
    \end{itemize}
    Cost of factorization: $\tildO(rs^{(\omega+1)/2})$
\end{frame}

\begin{frame}\frametitle{Some history}
    \begin{enumerate}
    \item['91] Lenstra~\cite{LenstraJr91} proves that the isomorphism
      problem is in $\mathsf{P}$.
      \begin{itemize}
      \item Based on Kummer theory, pervasive use of linear algebra.
      \item Does not prove precise complexity. Rough estimate:
        $\Omega(r^3)$.
      \end{itemize}
    \item['92] Pinch's algorithm~\cite{Pinch}:
      \begin{itemize}
      \item Based on mapping algebraic groups over $k,K$.
      \item Incomplete algorithm, no complexity analysis.
      \end{itemize}
    \item['96] Rains~\cite{rains2008} generalizes Pinch's algorithm.
      \begin{itemize}
      \item Complete algorithm, rigorous complexity analysis.
      \item Unpublished. Leaves open question of using elliptic
        curves.
      \end{itemize}
    \item['97] Magma~\cite{MAGMA} implements lattices of finite fields
      using on polynomial factorization and linear algebra~\cite{bosma+cannon+steel97}.
    \end{enumerate}
  \end{frame}
\begin{frame}\frametitle{Some history (cont.)}
  \begin{enumerate}
    \item['02] Allombert's variant of Lenstra's algorithm~\cite{Allombert02,Allombert02-rev}:
      \begin{itemize}
      \item Trades determinism for efficiency.
      \item Implementation integrated into Pari/GP~\cite{Pari}.
      \end{itemize}
    \item['07] Magma implements Rains' algorithm.
    \item['16] Narayanan proves the first $\tildO(r^2)$ upper
      bound~\cite{narayanan2016fast}.
      \begin{itemize}
      \item Variant of Allombert's algorithm.
      \item Using asymptotically fast modular composition.
      \end{itemize}
    \item[Now] Knowledge systematization. Notable results:
      \begin{itemize}
      \item Better variants of Allombert's algorithm.
      \item $\tildO(r^2)$ upper bound \emph{without fast modular
          composition}.
      \item Generalized Rains' algorithm to elliptic curves.
      \item C/Flint~\cite{hart2010flint} and Sage~\cite{Sage} implementations, experiments, comparisons.
      \end{itemize}
    \end{enumerate}
  \end{frame}

\subsection{Using Kummer theory}

\begin{frame}\frametitle{Allombert's algorithm}
      \begin{wrapfigure}[9]{r}{0.35\textwidth}
      \begin{tikzpicture}[node distance=5em,on grid,black!70]
        \node(Fq){$\F_q$};
        \node(Fqz)[above=of Fq]{$\F_q(\zeta)$};
        \node(k)[above left=of Fq]{$k$};
        \node(kz)[above=of k]{$k\otimes\F_q(\zeta)$};
        \node(K)[above right=of Fq]{$K$};
        \node(Kz)[above=of K]{$K\otimes\F_q(\zeta)$};
        \draw[font=\small,above]
        (Fq) edge (Fqz)
        (Fq) edge node{$\sigma$} (k)
        (Fqz) edge node{$\sigma$} (kz)
        (k) edge (kz)
        (Fq) edge node{$\sigma$} (K)
        (Fqz) edge node{$\sigma$} (Kz)
        (K) edge (Kz)
        (kz) edge node{$\sim$} (Kz);
      \end{tikzpicture}
    \end{wrapfigure}
    
    Assuming $\gcd(r,q)=1$:
    \begin{itemize}
    \item Let $h$ be an irreducible factor of the $r$-th cyclotomic
      polynomial over $\F_q$;
    \item Extend the action of $\Gal(k/\F_q)$ to the \textbf{ring}
      $k[\zeta]=k[Z]/h(Z)$:
      \[\begin{array}{llll}
          \sigma: & k[\zeta] & \rightarrow & k[\zeta], \\
                  & x \otimes \zeta & \mapsto & \sigma(x) \otimes \zeta;
        \end{array}\]
    \end{itemize}

    \begin{itemize}
    \item \textbf{Solve Hilbert 90:} find $\theta_1 \in k[\zeta]$ such that
      $\sigma(\theta_1) = \zeta\theta_1$ using linear algebra;
    \item Compute $\theta_2 \in K[\zeta]$ similarly;
    \end{itemize}

    \begin{itemize}
    \item Compute $c = \sqrt[r]{\theta_1^r/\theta_2^r} \in\F_q(\zeta)$;
    \item Project $\theta_1\mapsto\alpha\in k$ and
      $c\theta_2\mapsto\beta\in K$.
    \end{itemize}
  \end{frame}

\begin{frame}\frametitle{Implementation (take 1)}
  \begin{block}{Factorization}
  \begin{itemize}
  \item The factor $h$ of $\Phi_r$ is of degree $\ord_r(q) = O(r)$;
  \item Computing it is $\tildO(r)$ using Shoup~\cite{shoup94};
  \item Computing $r$-th roots in $\F_q(\zeta)$ is $\tildO(r^2)$ using Kaltofen--Shoup~\cite{kaltofen+shoup97}.
  \end{itemize}
\end{block}
  \begin{block}{Linear algebra}
  \begin{itemize}
  \item Computing a matrix for $\sigma$ over $\F_q$ is $\tildO(r^2)$;
  \item Computing its kernel over $\F_q(\zeta)$ is \alert{$\tildO((sr)^\omega)$}.
  \end{itemize}
\end{block}
\end{frame}

\begin{frame}\frametitle{Implementation (take 2, 3, 4, 5, \ldots)}
  \begin{block}{Reaching subquadratic complexity}
    \begin{enumerate}
    \item Use the factorization $h(S) = (S - \zeta) b(S)$ to perform linear algebra over $\F_q$.
    \item Use the factorization $S^r - 1 =  (S - \zeta) b(S) g(S)$ with $h$ and $g$ in $\F_q[S]$ to replace linear algebra by modular composition.
    \end{enumerate}
\end{block}
\end{frame}

\begin{frame}
%  \begin{figure}
    \centering
    \includegraphics[width=.8\textwidth]{../../paper/plots/bench-allombert-lowaux}
%    \caption{
      \flushleft
      \textbf{Allombert's algorithm} where the auxiliary degree
      $s=\ord_r(q)\le 10$.  Dots represent individual runs, lines
      represent degree 2 linear regressions.
%    }
%  \end{figure}
\end{frame}

\begin{frame}
%  \begin{figure}
    \centering
    \includegraphics[width=.8\textwidth]{../../paper/plots/bench-allombert-anyaux}
%    \caption{
      \flushleft
      \textbf{Allombert's algorithm,} as a function of the auxiliary degree
      $s=\ord_r(q)$ scaled down by $r^2$.
%    }
%  \end{figure}
\end{frame}

\subsection{Using algebraic groups}

\begin{frame}\frametitle{Pinch's algorithm}
      \begin{wrapfigure}[3]{r}{0pt}
      \begin{tikzpicture}[node distance=6em,font=\Large,on grid]
        \node(Fq){$\F_q$};
        \node(kum)[above=of Fq]{$\F_q[\mu_\ell]$};
        \node(k)[left=4em of kum]{$k$};
        \node(K)[right=4em of kum]{$K$};
        \draw[auto,font=\small]
        (Fq) edge (kum)
        (Fq) edge (k)
        (Fq) edge (K)
        (k) edge node{$\sim$} (kum)
        (kum) edge node{$\sim$} (K);
      \end{tikzpicture}
    \end{wrapfigure}

    \paragraph{Pinch's idea}
    \begin{itemize}
    \item Find \emph{small} $\ell$ such that $k\simeq\F_q[\mu_\ell]$,
    \item Pick $\ell$-th roots of unity $\alpha\in k$, $\beta\in K$,
    \item Find $e$ s.t. $\alpha\mapsto\beta^e$ using brute force.
    \item \textbf{Problem 1:} worst case $\ell\in O(q^r)$.
    \item \textbf{Problem 2:} potentially $O(\ell)$ exponents $e$ to test
      depending on the splitting of $\Phi_l$ over $\F_q$.
    \end{itemize}
\end{frame}

\begin{frame}\frametitle{Rains' algorithm and variants}
    \begin{itemize}
    \item Replace $\alpha,\beta$ with \emph{Gaussian periods}:
      \[\eta(\alpha) = \sum_{\sigma\in S}\alpha^\sigma\]
      where $(\Z/\ell\Z)^\times = \langle q\rangle \times S$.
    \begin{itemize}
    \item Periods are normal elements, hence yield bases of $k$, $K$.
    \item Periods are unique up to Galois action, hence $\eta(\alpha)\mapsto\eta(\beta)$ always defines an isomorphism.
      \item The size of $\ell$ can be controlled by allowing auxiliary extensions.
    \end{itemize}
    \end{itemize}
    \begin{itemize}
    \item Use higher dimensional algebraic groups:
  \begin{itemize}
  \item Replace $\F_q[\mu_\ell]$ with the $\ell$-torsion of
    \emph{random} elliptic curves $E/\F_q$;
  \item Replace Gaussian periods with \emph{elliptic
      periods}~\cite{mihailescu+morain+schost07};
  \item This removes the need for auxiliary extensions.
  \end{itemize}
      \end{itemize}
\end{frame}

\begin{frame}
  %\begin{figure}
    \centering
    \includegraphics[width=.75\textwidth]{../../paper/plots/bench-all}
    % \caption{
    \flushleft
      \textbf{Allombert's vs Rains'} at some fixed auxiliary extension
      degrees $s$. Lines represent median times, shaded areas minimum
      and maximum times.
   % }
%  \end{figure}
\end{frame}

%%%%%%%%%%%%%%%%%%%%%%%%%%%%%%%%%%%%%%%%%%%%%%%%%%%%%%%%%%%%%%%%%%%%%%
%%%%%%%%%%%%%%%%%%%%%%%%%%%%%%%%%%%%%%%%%%%%%%%%%%%%%%%%%%%%%%%%%%%%%%

\tikzset{round/.style={circle, draw=black, very thick, scale = 0.7}}
\tikzset{arrow/.style={->, >=latex}}
\tikzset{dashed-arrow/.style={->, >=latex, dashed}}
\newcommand{\ie}{\emph{i.e. }}
\newcommand{\eg}{\emph{e.g. }}
\newcommand{\embed}[2]{\phi_{#1\hookrightarrow#2}}

\begin{frame}
  \centering
  \huge{Part II: Compatible embeddings}
\end{frame}

\begin{frame}{The compatibility problem}
  \textcolor{purple}{\textbf{Context:}}
  \begin{itemize}
    \item $E$, $F$, $G$ fields
    \item $E$ subfield of $F$ and $F$ subfield of $G$
    \item $\phi_{E\hookrightarrow F}$, $\embed{F}{G}$,
      $\embed{E}{G}$ embeddings
  \end{itemize}
  \begin{figure}
    \centering
    \begin{tikzpicture}
      \node (E) at (0, 0) {$E$}; 
      \node (F) at (1.5, 1) {$F$}; 
      \node (G) at (0.5, 2) {$G$}; 

      \draw[arrow] (E) -- (F);
      \draw[arrow] (E) -- (G);
      \draw[arrow] (F) -- (G);

      \node (f12) at (1.25, 0.25) {$\embed{E}{F}$};
      \node (f13) at (-0.35, 1) {$\embed{E}{G}$};
      \node (f23) at (1.6, 1.65) {$\embed{F}{G}$};
    \end{tikzpicture}
  \end{figure}
  \[
    \textcolor{purple}{\embed{F}{G}\circ\embed{E}{F}\overset{?}{=}\embed{E}{G}}
  \]
\end{frame}

\begin{frame}{The compatibility problem}
   \begin{figure}
    \centering
    \begin{tikzpicture}
      \node (E) at (0, 0) {$E$}; 
      \node (F) at (2, 2) {$F$}; 
      \node (G) at (-2, 2) {$G$}; 
      \node (H) at (0, 4) {$H$}; 

      \draw[arrow] (E) -- (F);
      \draw[arrow] (E) -- (G);
      \draw[arrow] (G) -- (H);
      \draw[arrow] (F) -- (H);

      \node (f12) at (1.65, 0.8) {$\embed{E}{F}$};
      \node (f13) at (-1.65, 0.8) {$\embed{E}{G}$};
      \node (f24) at (1.75, 3) {$\embed{F}{G}$};
      \node (f34) at (-1.65, 3.2) {$\embed{G}{H}$};
    \end{tikzpicture}
  \end{figure}
  \[
    \textcolor{purple}{\embed{G}{H}\circ\embed{E}{G}\overset{?}{=}\embed{F}{H}\circ\embed{E}{F}}
  \]
\end{frame}

\begin{frame}{Bosma, Cannon and Steel '97~\cite{bosma+cannon+steel97}}
  \begin{itemize}
    \item Based upon \emph{naive} embedding algorithms.
    \item Supports arbitrary, user-defined finite fields.
    \item Allows to compute the embeddings in arbitrary order.
    \item Implemented by MAGMA.
  \end{itemize}
\end{frame}

\begin{frame}{The Bosma, Cannon and Steel framework}
  \begin{figure}
    \centering
    \begin{tikzpicture}
      \node (E) at (0, 0) {$E$}; 
      \node (F) at (1.5, 1) {$F$}; 
      \node (G) at (0.5, 2) {$G$}; 

      \draw[arrow] (E) -- (F);
      \draw[arrow] (E) -- (G);
      \draw[dashed-arrow] (F) -- (G);

      \node (f12) at (1.25, 0.25) {$\embed{E}{F}$};
      \node (f13) at (-0.35, 1) {$\embed{E}{G}$};
    \end{tikzpicture}
  \end{figure}
  \begin{itemize}
    \item Take $\embed{F}{G}'$ an arbitrary embedding between $F$ and
      $G$
    \item Find $\sigma\in\Gal(G/\mathbb{F}_p)$ such that
      $\sigma\circ\embed{F}{G}'\circ\embed{E}{F}=\embed{E}{G}$
    \item Set $\embed{F}{G}:=\sigma\circ\embed{F}{G}'$
    \item There are $|\Gal(F/E)|$ compatible morphisms
  \end{itemize}
\end{frame}

\begin{frame}{Bosma, Cannon and Steel framework}
  What about several subfields $E_1, E_2, \dots, E_r$ ?
  \begin{itemize}
    \item Enforce these axioms on the lattice:
      \begin{itemize}
        \item[CE1] (Unicity) At most one morphism $\embed{E}{F}$
        \item[CE2] (Reflexivity) For each $E$, $\embed{E}{E}=\Id_E$
        \item[CE3] (Invertibility) For each pair $(E, F)$ with $E\cong F$,
          $\embed{E}{F}=\embed{F}{E}^{-1}$ 
        \item[CE4] (Transitivity) For any triple $(E, F, G)$ with $E$
          subfield of $F$ and $F$ subfield of $G$, if we have
         computed $\embed{E}{F}$ and $\embed{F}{G}$, then
         $\embed{E}{G}=\embed{F}{G}\circ\embed{E}{F}$
       \item[CE5] (\textbf{Intersections}) For any triple $(E, F, G)$
         with $E$ and $F$ subfields of $G$, we have that the field
         $S=E\cap F$ is embedded in $E$ and $F$, \ie we have computed
         $\embed{S}{E}$ and $\embed{S}{F}$
      \end{itemize}
  \end{itemize}
\end{frame}

\begin{frame}{The Bosma, Cannon and Steel framework}
    \begin{figure}
    \centering
    \begin{tikzpicture}
      \node (E1) at (-2, 0) {$E_1$}; 
      \node (E2) at (-1, 0) {$E_2$}; 
      \node (Er) at (0.75, 0) {$E_r$}; 
      \node (F) at (1.5, 1) {$F$}; 
      \node (G) at (0.5, 2) {$G$}; 
      \node (p) at (0, 0) {$\dots$};

      \draw[arrow] (E1) -- (F);
      \draw[arrow] (E1) -- (G);
      \draw[arrow] (E2) -- (F);
      \draw[arrow] (E2) -- (G);
      \draw[arrow] (Er) -- (F);
      \draw[arrow] (Er) -- (G);
      \draw[dashed-arrow] (F) -- (G);
    \end{tikzpicture}
  \end{figure}
  \begin{itemize}
    \item Set $F'$ the field generated by the fields $E_i$ in $F$
    \item Set $G'$ the field generated by the fields $E_i$ in $G$
  \end{itemize}
  \begin{block}{Theorem}
There exists a unique isomorphism $\chi:F'\to G'$ that is compatible with all
embeddings, \ie such that for all $i$,
$\embed{E_i}{G'}=\chi\circ\embed{E_i}{F'}$.
  \end{block}
\end{frame}


\begin{frame}{New problem: compute embeddings with common subfields}
  \begin{itemize}
    \item We want to embed $E$ in $F$
      \begin{itemize}
        \item additionnal information: $S$ is a field embedded in $E$ and $F$
      \end{itemize}
    \item The naive algorithm can be sped up by replacing $\F_p$ with
      $S$ as base field\\
      (degree $[E:S]$ polynomial factorization \textbf{vs} degree
      $[E:\mathbb{F}_p]$)
    \item More generally: $S$ the compositum of all \textbf{known}
      fields embedded in $E$ and $F$.
  \end{itemize}
\end{frame}

\begin{frame}{Some questions}
  \begin{itemize}
  \item Bosma, Cannon and Steel framework + Allombert's algorithm:\\
    \emph{any smart optimizations possible}?
  \item Allombert's algorithm\\ \emph{with common subfield knowledge}?
  \end{itemize}
\end{frame}

\begin{frame}{Demo}
  \begin{itemize}
  \item Our implementations of Allombert's algorithm + \emph{embedding
      evaluation} are being pushed into Flint
    (\url{https://github.com/wbhart/flint2/pull/351});
  \item A compatible embedding framework is being added to
    \href{http://nemocas.org/}{Nemo}
    (\url{https://github.com/Nemocas/Nemo.jl/issues/233}).
  \end{itemize}
\end{frame}

\begin{frame}
\Huge \textbf{Questions ?}
\end{frame}

\begin{frame}[allowframebreaks]\frametitle{\refname}
  \scriptsize
    \bibliographystyle{plain}
    \bibliography{../../paper/refs}
\end{frame}

\end{document}
